\chapter{{\tt PatchAndGoInfo}: Pivot Modification Object}
\par
On occasion, an application will demand specific behavior 
during a factorization.
We have written the {\tt PatchAndGoInfo} object to communicate
information to the {\tt Chv} object during a factorization of a front.
Most users can ignore this object.
However, if a different type of behavior is required, one could
extend this object by adding a new strategy to it and modifying the
{\tt Chv} methods that factor a front.
\par
Let us describe two strategies that we presently support.
\begin{itemize}
\item
Primal-dual linear programming may require repeated factorizations
of matrices of the form $AD^2A^T$, where $A$ comes from constraint
equations and $D$ is a diagonal matrix.
As the optimization proceeds,
$AD^2A^T$ becomes increasingly ill-conditioned because the
entries in $D$ go to zero or infinity.
Normally, when a small or zero pivot element is detected, we would
either signal an error (if we expected the matrix to be positive
definite) or pivot for stability.
However, in the primal-dual pivot context, a small or zero element
on the diagonal is not a calamity.
It signals that the variable associated with the small entry can be
``skipped'' in the solution process.
There are several ways to implement this behavior.
We have chosen a simple way: the diagonal entry is set to 1.0
and all off-diagonal entries in the corresponding column of $L$
are set to zero.
\item
In structural analysis, ``multi-point constraints'' are often
applied to a linear system. At times, applying these constraints
generates a matrix that is essentially singular. The singularity
may be benign, as in the following case.
$$
\left \lbrack \begin{array}{cc}
A_{1,1} & 0 \\
0 & A_{2,2}
\end{array} \right \rbrack
\left \lbrack \begin{array}{c}
X_1 \\
X_2
\end{array} \right \rbrack
=
\left \lbrack \begin{array}{c}
0 \\
B_2
\end{array} \right \rbrack
$$
If $A_{1,1}$ is singular, the solution $X_1 = 0$ and
$X_2 = A_{2,2}^{-1} B_2$ is perfectly acceptable.
In other cases, the location of the singularity can be
communicated back to the user to supply useful information
about the finite element model.
One common practice is to not use pivoting, but to check the 
magnitude of the diagonal entry as a row and column is to be eliminated.
If the magnitude is smaller than a user-supplied parameter,
the diagonal entry is set to some multiple of the largest
offdiagonal entry in that row and column of the front,
the location and perturbation is noted, 
and the factorization proceeds.
\end{itemize}
\par
Other strategies can be added to the {\tt PatchAndGoInfo} object.
For example, if a matrix is being factored that is believed to be
positive definite, and a negative value is found in a pivot
element, one could abort the factorization, or perturb the element
so that it is positive.
