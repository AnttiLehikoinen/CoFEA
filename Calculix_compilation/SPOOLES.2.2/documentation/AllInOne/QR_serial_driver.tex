\vfill \eject
\section{{\tt allInOne.c} -- A Serial $QR$ Driver Program}
\label{section:QR-serial-driver}

\begin{verbatim}
/*  QRallInOne.c  */

#include "../../misc.h"
#include "../../FrontMtx.h"
#include "../../SymbFac.h"

/*--------------------------------------------------------------------*/
int
main ( int argc, char *argv[] ) {
/*
   --------------------------------------------------
   QR all-in-one program
   (1) read in matrix entries and form InpMtx object
       of A and A^TA
   (2) form Graph object of A^TA
   (3) order matrix and form front tree
   (4) get the permutation, permute the matrix and 
       front tree and get the symbolic factorization
   (5) compute the numeric factorization
   (6) read in right hand side entries
   (7) compute the solution

   created -- 98jun11, cca
   --------------------------------------------------
*/
/*--------------------------------------------------------------------*/
char            *matrixFileName, *rhsFileName ;
ChvManager      *chvmanager ;
DenseMtx        *mtxB, *mtxX ;
double          facops, imag, real, value ;
double          cpus[10] ;
ETree           *frontETree ;
FILE            *inputFile, *msgFile ;
FrontMtx        *frontmtx ;
Graph           *graph ;
int             ient, irow, jcol, jrhs, jrow, msglvl, neqns,
                nedges, nent, nrhs, nrow, seed, type ;
InpMtx          *mtxA ;
IV              *newToOldIV, *oldToNewIV ;
IVL             *adjIVL, *symbfacIVL ;
SubMtxManager   *mtxmanager ;
/*--------------------------------------------------------------------*/
/*
   --------------------
   get input parameters
   --------------------
*/
if ( argc != 7 ) {
   fprintf(stdout, 
      "\n usage: %s msglvl msgFile type matrixFileName rhsFileName seed"
      "\n    msglvl -- message level"
      "\n    msgFile -- message file"
      "\n    type    -- type of entries"
      "\n      1 (SPOOLES_REAL)    -- real entries"
      "\n      2 (SPOOLES_COMPLEX) -- complex entries"
      "\n    matrixFileName -- matrix file name, format"
      "\n       nrow ncol nent"
      "\n       irow jcol entry"
      "\n        ..."
      "\n        note: indices are zero based"
      "\n    rhsFileName -- right hand side file name, format"
      "\n       nrow "
      "\n       entry[0]"
      "\n       ..."
      "\n       entry[nrow-1]"
      "\n    seed -- random number seed, used for ordering"
      "\n", argv[0]) ;
   return(0) ;
}
msglvl = atoi(argv[1]) ;
if ( strcmp(argv[2], "stdout") == 0 ) {
   msgFile = stdout ;
} else if ( (msgFile = fopen(argv[2], "a")) == NULL ) {
   fprintf(stderr, "\n fatal error in %s"
           "\n unable to open file %s\n",
           argv[0], argv[2]) ;
   return(-1) ;
}
type           = atoi(argv[3]) ;
matrixFileName = argv[4] ;
rhsFileName    = argv[5] ;
seed           = atoi(argv[6]) ;
/*--------------------------------------------------------------------*/
/*
   --------------------------------------------
   STEP 1: read the entries from the input file 
   and create the InpMtx object of A
   --------------------------------------------
*/
inputFile = fopen(matrixFileName, "r") ;
fscanf(inputFile, "%d %d %d", &nrow, &neqns, &nent) ;
mtxA = InpMtx_new() ;
InpMtx_init(mtxA, INPMTX_BY_ROWS, type, nent, 0) ;
if ( type == SPOOLES_REAL ) {
   for ( ient = 0 ; ient < nent ; ient++ ) {
      fscanf(inputFile, "%d %d %le", &irow, &jcol, &value) ;
      InpMtx_inputRealEntry(mtxA, irow, jcol, value) ;
   }
} else {
   for ( ient = 0 ; ient < nent ; ient++ ) {
      fscanf(inputFile, "%d %d %le %le", &irow, &jcol, &real, &imag) ;
      InpMtx_inputComplexEntry(mtxA, irow, jcol, real, imag) ;
   }
}
fclose(inputFile) ;
if ( msglvl > 1 ) {
   fprintf(msgFile, "\n\n input matrix") ;
   InpMtx_writeForHumanEye(mtxA, msgFile) ;
   fflush(msgFile) ;
}
/*--------------------------------------------------------------------*/
/*
   ----------------------------------------
   STEP 2: read the right hand side entries
   ----------------------------------------
*/
inputFile = fopen(rhsFileName, "r") ;
fscanf(inputFile, "%d %d", &nrow, &nrhs) ;
mtxB = DenseMtx_new() ;
DenseMtx_init(mtxB, type, 0, 0, nrow, nrhs, 1, nrow) ;
DenseMtx_zero(mtxB) ;
if ( type == SPOOLES_REAL ) {
   for ( irow = 0 ; irow < nrow ; irow++ ) {
      fscanf(inputFile, "%d", &jrow) ;
      for ( jrhs = 0 ; jrhs < nrhs ; jrhs++ ) {
         fscanf(inputFile, "%le", &value) ;
         DenseMtx_setRealEntry(mtxB, jrow, jrhs, value) ;
      }
   }
} else {
   for ( irow = 0 ; irow < nrow ; irow++ ) {
      fscanf(inputFile, "%d", &jrow) ;
      for ( jrhs = 0 ; jrhs < nrhs ; jrhs++ ) {
         fscanf(inputFile, "%le %le", &real, &imag) ;
         DenseMtx_setComplexEntry(mtxB, jrow, jrhs, real, imag) ;
      }
   }
}
fclose(inputFile) ;
if ( msglvl > 1 ) {
   fprintf(msgFile, "\n\n rhs matrix in original ordering") ;
   DenseMtx_writeForHumanEye(mtxB, msgFile) ;
   fflush(msgFile) ;
}
/*--------------------------------------------------------------------*/
/*
   -------------------------------------------------
   STEP 3 : find a low-fill ordering
   (1) create the Graph object for A^TA or A^HA
   (2) order the graph using multiple minimum degree
   -------------------------------------------------
*/
graph = Graph_new() ;
adjIVL = InpMtx_adjForATA(mtxA) ;
nedges = IVL_tsize(adjIVL) ;
Graph_init2(graph, 0, neqns, 0, nedges, neqns, nedges, adjIVL,
            NULL, NULL) ;
if ( msglvl > 1 ) {
   fprintf(msgFile, "\n\n graph of A^T A") ;
   Graph_writeForHumanEye(graph, msgFile) ;
   fflush(msgFile) ;
}
frontETree = orderViaMMD(graph, seed, msglvl, msgFile) ;
if ( msglvl > 1 ) {
   fprintf(msgFile, "\n\n front tree from ordering") ;
   ETree_writeForHumanEye(frontETree, msgFile) ;
   fflush(msgFile) ;
}
/*--------------------------------------------------------------------*/
/*
   -----------------------------------------------------
   STEP 4: get the permutation, permute the matrix and 
           front tree and get the symbolic factorization
   -----------------------------------------------------
*/
oldToNewIV = ETree_oldToNewVtxPerm(frontETree) ;
newToOldIV = ETree_newToOldVtxPerm(frontETree) ;
InpMtx_permute(mtxA, NULL, IV_entries(oldToNewIV)) ;
InpMtx_changeStorageMode(mtxA, INPMTX_BY_VECTORS) ;
symbfacIVL = SymbFac_initFromGraph(frontETree, graph) ;
IVL_overwrite(symbfacIVL, oldToNewIV) ;
IVL_sortUp(symbfacIVL) ;
ETree_permuteVertices(frontETree, oldToNewIV) ;
if ( msglvl > 1 ) {
   fprintf(msgFile, "\n\n old-to-new permutation vector") ;
   IV_writeForHumanEye(oldToNewIV, msgFile) ;
   fprintf(msgFile, "\n\n new-to-old permutation vector") ;
   IV_writeForHumanEye(newToOldIV, msgFile) ;
   fprintf(msgFile, "\n\n front tree after permutation") ;
   ETree_writeForHumanEye(frontETree, msgFile) ;
   fprintf(msgFile, "\n\n input matrix after permutation") ;
   InpMtx_writeForHumanEye(mtxA, msgFile) ;
   fprintf(msgFile, "\n\n symbolic factorization") ;
   IVL_writeForHumanEye(symbfacIVL, msgFile) ;
   fflush(msgFile) ;
}
/*--------------------------------------------------------------------*/
/*
   ------------------------------------------
   STEP 5: initialize the front matrix object
   ------------------------------------------
*/
frontmtx = FrontMtx_new() ;
mtxmanager = SubMtxManager_new() ;
SubMtxManager_init(mtxmanager, NO_LOCK, 0) ;
if ( type == SPOOLES_REAL ) {
   FrontMtx_init(frontmtx, frontETree, symbfacIVL, type, 
                 SPOOLES_SYMMETRIC, FRONTMTX_DENSE_FRONTS, 
                 SPOOLES_NO_PIVOTING, NO_LOCK, 0, NULL,
                 mtxmanager, msglvl, msgFile) ;
} else {
   FrontMtx_init(frontmtx, frontETree, symbfacIVL, type, 
                 SPOOLES_HERMITIAN, FRONTMTX_DENSE_FRONTS, 
                 SPOOLES_NO_PIVOTING, NO_LOCK, 0, NULL,
                 mtxmanager, msglvl, msgFile) ;
}
/*--------------------------------------------------------------------*/
/*
   -----------------------------------------
   STEP 6: compute the numeric factorization
   -----------------------------------------
*/
chvmanager = ChvManager_new() ;
ChvManager_init(chvmanager, NO_LOCK, 1) ;
DVzero(10, cpus) ;
facops = 0.0 ;
FrontMtx_QR_factor(frontmtx, mtxA, chvmanager, 
                   cpus, &facops, msglvl, msgFile) ;
ChvManager_free(chvmanager) ;
if ( msglvl > 1 ) {
   fprintf(msgFile, "\n\n factor matrix") ;
   fprintf(msgFile, "\n facops = %9.2f", facops) ;
   FrontMtx_writeForHumanEye(frontmtx, msgFile) ;
   fflush(msgFile) ;
}
/*--------------------------------------------------------------------*/
/*
   --------------------------------------
   STEP 7: post-process the factorization
   --------------------------------------
*/
FrontMtx_postProcess(frontmtx, msglvl, msgFile) ;
if ( msglvl > 1 ) {
   fprintf(msgFile, "\n\n factor matrix after post-processing") ;
   FrontMtx_writeForHumanEye(frontmtx, msgFile) ;
   fflush(msgFile) ;
}
/*--------------------------------------------------------------------*/
/*
   -------------------------------
   STEP 8: solve the linear system
   -------------------------------
*/
mtxX = DenseMtx_new() ;
DenseMtx_init(mtxX, type, 0, 0, neqns, nrhs, 1, neqns) ;
FrontMtx_QR_solve(frontmtx, mtxA, mtxX, mtxB, mtxmanager,
                  cpus, msglvl, msgFile) ;
if ( msglvl > 1 ) {
   fprintf(msgFile, "\n\n solution matrix in new ordering") ;
   DenseMtx_writeForHumanEye(mtxX, msgFile) ;
   fflush(msgFile) ;
}
/*--------------------------------------------------------------------*/
/*
   -------------------------------------------------------
   STEP 9: permute the solution into the original ordering
   -------------------------------------------------------
*/
DenseMtx_permuteRows(mtxX, newToOldIV) ;
if ( msglvl > 0 ) {
   fprintf(msgFile, "\n\n solution matrix in original ordering") ;
   DenseMtx_writeForHumanEye(mtxX, msgFile) ;
   fflush(msgFile) ;
}
/*--------------------------------------------------------------------*/
/*
   ------------------------
   free the working storage
   ------------------------
*/
InpMtx_free(mtxA) ;
FrontMtx_free(frontmtx) ;
Graph_free(graph) ;
DenseMtx_free(mtxX) ;
DenseMtx_free(mtxB) ;
ETree_free(frontETree) ;
IV_free(newToOldIV) ;
IV_free(oldToNewIV) ;
IVL_free(symbfacIVL) ;
SubMtxManager_free(mtxmanager) ;
/*--------------------------------------------------------------------*/
return(1) ; }
/*--------------------------------------------------------------------*/
\end{verbatim}
