\par
\subsection{{\tt PCV} : {\tt char *} vector methods}
\label{subsection:Utilities:proto:PCV}
\par
%=======================================================================
\begin{enumerate}
%-----------------------------------------------------------------------
\item
\begin{verbatim}
char ** PCVinit ( int n ) ;
\end{verbatim}
\index{PCVinit@{\tt PCVinit()}}
This is the allocator and initializer method for {\tt char*} vectors.
Storage for an array with size {\tt n} is found and each
entry is filled with {\tt NULL}.
A pointer to the array is returned.
%-----------------------------------------------------------------------
\item
\begin{verbatim}
void PCVfree ( char **p_vec ) ;
\end{verbatim}
\index{PCVfree@{\tt PCVfree()}}
This method releases the storage taken by {\tt p\_vec[]}.
%-----------------------------------------------------------------------
\item
\begin{verbatim}
void PCVcopy ( int n, char *p_y[], char *p_x[] ) ;
\end{verbatim}
\index{PCVcopy@{\tt PCVcopy()}}
This method copies {\tt n} entries from {\tt p\_x[]} to {\tt p\_y[]},
i.e.,
{\tt p\_y[i] = p\_x[i]} for {\tt 0 <= i < n}.
%-----------------------------------------------------------------------
\item
\begin{verbatim}
void PCVsetup ( int n, int sizes[], char vec[], char *p_vec[] ) ;
\end{verbatim}
\index{PCVsetup@{\tt PCVsetup()}}
This method sets the entries of {\tt p\_vec[]} as pointers into {\tt
vec[]} given by the {\tt sizes[]} vector,
i.e.,
{\tt p\_vec[0] = vec}, and 
{\tt p\_vec[i] = p\_vec[i-1] + sizes[i-1]} 
for {\tt 0 < i < n}.
%-----------------------------------------------------------------------
\end{enumerate}
