\par
\section{Driver programs}
\label{section:MPI:drivers}
\par
%=======================================================================
\begin{enumerate}
%-----------------------------------------------------------------------
\item
\begin{verbatim}
allInOne msglvl msgFile type symmetryflag pivotingflag seed
\end{verbatim}
This driver program is an example program for reading in a linear
system and right hand side, ordering the matrix, 
factoring the matrix, and solving the system.
Use the script file {\tt do\_AllInOne} for testing.
\par
The files names for the matrix and right hand side entries are
hardcoded.
Processor $q$ reads in matrix entries from file 
$\mbox{\tt matrix.}q\mbox{\tt .input}$
and right hand side entries from file
$\mbox{\tt rhs.}q\mbox{\tt .input}$.
The format for the matrix files is as follows:
\begin{verbatim}
neqns neqns nent
irow jcol entry
...  ...  ...
\end{verbatim}
where {\tt neqns} is the global number of equations and {\tt nent}
is the number of entries in this file.
There follows {\tt nent} lines, each containing a row index, a
column index and one or two floating point numbers, one if real,
two if complex.
The format for the right hand side file is similar:
\begin{verbatim}
nrow nrhs
irow entry ... entry
...  ...   ... ...
\end{verbatim}
where {\tt nrow} is the number of rows in this file
and {\tt nrhs} is the number of rigght and sides.
There follows {\tt nrow} lines, each containing a row index
and either {\tt nrhs} or {\tt 2*nrhs} floating point numbers, 
the first if real, the second if complex.
\par
\begin{itemize}
\item
The {\tt msglvl} parameter determines the amount of output.
Use {\tt msglvl = 1} for just timing output.
\item
The {\tt msgFile} parameter determines the message file --- if {\tt
msgFile} is {\tt stdout}, then the message file is {\it stdout},
otherwise a file is opened with {\it append} status to receive any
output data.
\item
The {\tt type} parameter specifies whether the linear system is
real ({\tt type} = 1) or complex ({\tt type} = 2).
\item
The {\tt symmetryflag} parameter specifies whether the matrix is
symmetric ({\tt symmetryflag} = 0),
Hermitian ({\tt symmetryflag} = 1) or
nonsymmetric ({\tt symmetryflag} = 2)
\item
The {\tt pivotingflag} parameter specifies whether pivoting will be
performed during the factorization,
yes ({\tt symmetryflag} = 0) or
no ({\tt symmetryflag} = 2).
The pivot tolerance is hardcoded as {\tt tau = 100.0}.
\item
The {\tt seed} parameter is a random number seed.
\end{itemize}
%-----------------------------------------------------------------------
\item
\begin{verbatim}
patchAndGoMPI msglvl msgFile type symmetryflag patchAndGoFlag 
              fudge toosmall storeids storevalues seed 
\end{verbatim}
This driver program is used to test the ``patch-and-go''
functionality for a factorization without pivoting.
When small diagonal pivot elements are found, 
one of three actions are taken.
See the {\tt PatchAndGoInfo} object for more information.
\par
The program reads in a matrix $A$ and right hand side $B$,
generates the graph for $A$ and orders the matrix,
factors $A$ and solves the linear system $AX = B$ for $X$.
\par
The files names for the matrix and right hand side entries are
hardcoded.
Processor $q$ reads in matrix entries from file 
$\mbox{\tt patchMatrix.}q\mbox{\tt .input}$
and right hand side entries from file
$\mbox{\tt patchRhs.}q\mbox{\tt .input}$.
The format for the matrix files is as follows:
\begin{verbatim}
neqns neqns nent
irow jcol entry
...  ...  ...
\end{verbatim}
where {\tt neqns} is the global number of equations and {\tt nent}
is the number of entries in this file.
There follows {\tt nent} lines, each containing a row index, a
column index and one or two floating point numbers, one if real,
two if complex.
The format for the right hand side file is similar:
\begin{verbatim}
nrow nrhs
irow entry ... entry
...  ...   ... ...
\end{verbatim}
where {\tt nrow} is the number of rows in this file
and {\tt nrhs} is the number of rigght and sides.
There follows {\tt nrow} lines, each containing a row index
and either {\tt nrhs} or {\tt 2*nrhs} floating point numbers, 
the first if real, the second if complex.
Use the script file {\tt do\_patchAndGo} for testing.
\par
\begin{itemize}
\item
The {\tt msglvl} parameter determines the amount of output.
Use {\tt msglvl = 1} for just timing output.
\item
The {\tt msgFile} parameter determines the message file --- if {\tt
msgFile} is {\tt stdout}, then the message file is {\it stdout},
otherwise a file is opened with {\it append} status to receive any
output data.
\item
The {\tt type} parameter specifies a real or complex linear system.
\begin{itemize}
\item
{\tt type = 1 (SPOOLES\_REAL)} for real,
\item
{\tt type = 2 (SPOOLES\_COMPLEX)} for complex.
\end{itemize}
\item
The {\tt symmetryflag} parameter specifies the symmetry of the matrix.
\begin{itemize}
\item
{\tt type = 0 (SPOOLES\_SYMMETRIC)} for $A$ real or complex symmetric,
\item
{\tt type = 1 (SPOOLES\_HERMITIAN)} for $A$ complex Hermitian,
\item
{\tt type = 2 (SPOOLES\_NONSYMMETRIC)}
\end{itemize}
for $A$ real or complex nonsymmetric.
\item
The {\tt patchAndGoFlag} specifies the ``patch-and-go'' strategy.
\begin{itemize}
\item
{\tt patchAndGoFlag = 0} --- if a zero pivot is detected, stop
computing the factorization, set the error flag and return.
\item
{\tt patchAndGoFlag = 1} --- if a small or zero pivot is detected,
set the diagonal entry to 1 and the offdiagonal entries to zero.
\item
{\tt patchAndGoFlag = 2} --- if a small or zero pivot is detected,
perturb the diagonal entry.
\end{itemize}
\item
The {\tt fudge} parameter is used to perturb a diagonal entry.
\item
The {\tt toosmall} parameter is judge when a diagonal entry is small.
\item
If {\tt storeids = 1}, then the locations where action was taken is
stored in an {\tt IV} object.
\item
If {\tt storevalues = 1}, then the perturbations are
stored in an {\tt DV} object.
\item
The {\tt seed} parameter is a random number seed.
\end{itemize}
%-----------------------------------------------------------------------
\item
\begin{verbatim}
testGather msglvl msgFile type nrow ncol inc1 inc2 seed
\end{verbatim}
This driver program test the {\tt DenseMtx\_MPI\_gatherRows()} method.
Each processor creates part of a distributed matrix $X$ and fills
its entries with entries known to all processors.
($X_{j,k} = j + k*\mbox{nproc}$ if real and
$X_{j,k} = j + k*\mbox{nproc} +
i*2*(j + k*\mbox{nproc})$) if complex.
The mapping from rows of $X$ to processors is random.
Each processor then generates a random vector that contains its rows
in a local $Y$, which will be filled with the corresponding rows of
$X$.
The rows of $X$ are then gathered into $Y$, and the local errors
are computed.
The global error is written to the results file by processor zero.
\par
Use the script file {\tt do\_gather} for testing.
\par
\begin{itemize}
\item
The {\tt msglvl} parameter determines the amount of output.
Use {\tt msglvl = 1} for just timing output.
\item
The {\tt msgFile} parameter determines the message file --- if {\tt
msgFile} is {\tt stdout}, then the message file is {\it stdout},
otherwise a file is opened with {\it append} status to receive any
output data.
\item
The {\tt type} parameter specifies whether the linear system is
real ({\tt type} = 1) or complex ({\tt type} = 2).
\item
{\tt nrow} is the number of rows in {\tt X}.
\item
{\tt ncol} is the number of columns in {\tt X}.
\item
{\tt inc1} is the row increment for {\tt X}.
\item
{\tt inc2} is the column increment for {\tt X}.
\item
The {\tt seed} parameter is a random number seed.
\end{itemize}
%-----------------------------------------------------------------------
\item
\begin{verbatim}
testGraph_Bcast msglvl msgFile type nvtx nitem root seed 
\end{verbatim}
This driver program tests the distributed 
{\tt Graph\_MPI\_Bcast()} method.
Processor {\tt root} generates a random graph of type {\tt type}
(see the documentation for the {\tt Graph} object in
chapter~\ref{chapter:Graph}) with {\tt nvtx} vertices.
The random graph is constructed via an {\tt InpMtx} object
using {\tt nitem} edges.
Processor {\tt root} then sends its {\tt Graph} object to the other
processors.
Each processor computes a checksum for its object, and the error
are collected on processor 0.
Use the script file {\tt do\_Graph\_Bcast} for testing.
\par
\begin{itemize}
\item
The {\tt msglvl} parameter determines the amount of output.
Use {\tt msglvl = 1} for just timing output.
\item
The {\tt msgFile} parameter determines the message file --- if {\tt
msgFile} is {\tt stdout}, then the message file is {\it stdout},
otherwise a file is opened with {\it append} status to receive any
output data.
\item
The {\tt type} parameter specifies the type of the graph,
unweighted, weighted vertices, weighted edges, and combinations.
\item
The {\tt nvtx} parameter specifies the number of vertices in the
graph.
\item
The {\tt nitem} parameter is used to specify the number of edges
that form the graph. An upper bound on the number of edges is
{\tt nvtx + 2*nitem}.
\item
{\tt root} is the root processor for the broadcast.
\item
The {\tt seed} parameter is a random number seed.
\end{itemize}
%-----------------------------------------------------------------------
\item
\begin{verbatim}
testGridMPI msglvl msgFile n1 n2 n3 maxzeros maxsize seed type
   symmetryflag sparsityflag pivotingflag tau droptol lookahead 
   nrhs maptype cutoff
\end{verbatim}
This driver program creates and solves the linear system $AX = Y$
where the structure of $A$ is from a 
${\tt n1} \times {\tt n2} \times {\tt n3}$ 
regular grid operator and is
ordered using nested dissection.
The front tree is formed allowing {\tt maxzeros} in a front with a
maximum of {\tt maxsize} vertices in a non-leaf front.
Process {\tt 0} generates the linear system and broadcasts the front
tree to the other processes.
Using {\tt maptype}, the processes generate the owners map 
for the factorization in parallel.
The $A$, $X$ and $Y$ matrices are then distributed among the
processes.
The symbolic factorization is then computed in parallel,
the front matrix is initialized, and the factorization is computed
in parallel.
If pivoting has taken place, the solution and right hand side
matrices are redistributed as necessary.
The matrix is post-processed where it is converted 
to a submatrix storage format.
Each processor computes the identical solve map, and the front
matrix is split among the processes.
The linear system is then solved in parallel and the error is
computed.
Use the script file {\tt do\_gridMPI} for testing.
\par
\begin{itemize}
\item
The {\tt msglvl} parameter determines the amount of output.
Use {\tt msglvl = 1} for just timing output.
\item
The {\tt msgFile} parameter determines the message file --- if {\tt
msgFile} is {\tt stdout}, then the message file is {\it stdout},
otherwise a file is opened with {\it append} status to receive any
output data.
\item
The {\tt n1} parameter is the number of grid points in the first
direction.
\item
The {\tt n2} parameter is the number of grid points in the second
direction.
\item
The {\tt n3} parameter is the number of grid points in the third
direction.
\item
The {\tt maxzeros} parameter is the maximum number of zero entries
allowed in a front.
\item
The {\tt maxsize} parameter is the maximum number of internal rows 
and columns allowed in a front.
\item
The {\tt seed} parameter is a random number seed.
\item
The {\tt type} parameter specifies whether the linear system is
real or complex.
Use {\tt 1} for real and {\tt 2} for complex.
\item
The {\tt symmetryflag} parameter denotes the presence or absence of
symmetry.
\begin{itemize}
\item
Use {\tt 0} for a real or complex symmetric matrix $A$.
A $(U^T + I)D(I + U)$ factorization is computed.
\item
Use {\tt 1} for a complex Hermitian matrix $A$.
A $(U^H + I)D(I + U)$ factorization is computed.
\item
Use {\tt 2} for a real or complex nonsymmetric matrix $A$.
A $(L + I)D(I + U)$ factorization is computed.
\end{itemize}
\item
The {\tt sparsityflag} parameter denotes a direct or approximate
factorization.
Valid values are {\tt 0} for a direct factorization
and {\tt 1} is for an approximate factorization.
\item
The {\tt pivotingflag} parameter denotes whether pivoting is to be
used in the factorization.
Valid values are {\tt 0} for no pivoting
and {\tt 1} to enable pivoting.
\item
The {\tt tau} parameter is used when pivoting is enabled,
in which case it is an upper bound on the magnitude of an entry in
the triangular factors $L$ and $U$.
\item
The {\tt droptol} parameter is used when an approximate
factorization is requested, in which it is a lower bound on the
magnitude of an entry in $L$ and $U$.
\item
The {\tt lookahead} parameter governs the
``upward--looking'' nature of the factorization.
Choosing {\tt lookahead = 0} is usually the most conservative with
respect to working storage, while positive values increase the
working storage and sometimes decrease the factorization time.
\item
The {\tt nrhs} parameter is the number of right hand sides.
\item
The {\tt maptype} parameter is the type of factorization map.
\begin{itemize}
\item {\tt 1} --- a wrap map via a post-order traversal
\item {\tt 2} --- a balanced map via a post-order traversal
\item {\tt 3} --- a subtree--subset map
\item {\tt 4} --- a domain decomposition map
\end{itemize}
\item
The {\tt cutoff} parameter is used with the domain decomposition
map, and specifies the maximum fraction of the vertices to be
included into a domain. Try {\tt cutoff} = {\tt 1/nproc}
or {\tt 1/(2*nproc)}.
\end{itemize}
%-----------------------------------------------------------------------
\item
\begin{verbatim}
testIV_allgather msglvl msgFile n seed 
\end{verbatim}
This driver program tests the distributed 
{\tt IV\_MPI\_allgather()} method.
Each processor generates the same {\tt owners[]} map
and fills an {\tt IV} object with random entries for the entries
which it owns.
The processors all-gather the entries of the vector so each
processor has a copy of the global vector.
Each processor computes a checksum of the vector to detect errors.
Use the script file {\tt do\_IVallgather} for testing.
\par
\begin{itemize}
\item
The {\tt msglvl} parameter determines the amount of output.
Use {\tt msglvl = 1} for just timing output.
\item
The {\tt msgFile} parameter determines the message file --- if {\tt
msgFile} is {\tt stdout}, then the message file is {\it stdout},
otherwise a file is opened with {\it append} status to receive any
output data.
\item
The {\tt n} parameter is the length of the vector.
\item
The {\tt seed} parameter is a random number seed.
\end{itemize}
%-----------------------------------------------------------------------
\item
\begin{verbatim}
testIVL_alltoall msglvl msgFile n seed
\end{verbatim}
This driver program tests the distributed {\tt IVL\_MPI\_alltoall()}
method.
This is used by the {\tt makeSendRecvIVLs} method when setting up
the distributed matrix-matrix multiply.
Each processor constructs a ``receive'' {\tt IVL} object with 
{\tt nproc} lists.
List {\tt iproc} contains a set of ids of items that this processor
will receive from processor {\tt iproc}.
The processors then call {\tt IVL\_MPI\_allgather} to create their
``send'' {\tt IVL} object,
where list {\tt iproc} contains a set of ids of items that 
this processor will send to processor {\tt iproc}.
The set of lists in all the ``receive'' {\tt IVL} objects is
exactly the same as the set of lists in all the ``send'' objects.
This is an ``all-to-all'' scatter/gather operation.
Had the lists be stored contiguously or at least in one block of
storage, we could have used the {\tt MPI\_Alltoallv()} method.
\par
Use the script file {\tt do\_IVL\_alltoall} for testing.
\par
\begin{itemize}
\item
The {\tt msglvl} parameter determines the amount of output.
Use {\tt msglvl = 1} for just timing output.
\item
The {\tt msgFile} parameter determines the message file --- if {\tt
msgFile} is {\tt stdout}, then the message file is {\it stdout},
otherwise a file is opened with {\it append} status to receive any
output data.
\item
The {\tt n} parameter is an upper bound on list size and element value.
\item
The {\tt seed} parameter is a random number seed.
\end{itemize}
%-----------------------------------------------------------------------
\item
\begin{verbatim}
testIVL_allgather msglvl msgFile nlist seed 
\end{verbatim}
This driver program tests the distributed 
{\tt IVL\_MPI\_allgather()} method.
Each processor generates the same {\tt owners[]} map
and fills an {\tt IVL} object with random entries for the lists
which it owns.
The processors all-gather the entries of the {\tt IVL} object so each
processor has a copy of the global object.
Each processor computes a checksum of the lists to detect errors.
Use the script file {\tt do\_IVLallgather} for testing.
\par
\begin{itemize}
\item
The {\tt msglvl} parameter determines the amount of output.
Use {\tt msglvl = 1} for just timing output.
\item
The {\tt msgFile} parameter determines the message file --- if {\tt
msgFile} is {\tt stdout}, then the message file is {\it stdout},
otherwise a file is opened with {\it append} status to receive any
output data.
\item
The {\tt nlist} parameter is the number of lists.
\item
The {\tt seed} parameter is a random number seed.
\end{itemize}
%-----------------------------------------------------------------------
\item
\begin{verbatim}
testIVL_Bcast msglvl msgFile nlist maxlistsize root seed 
\end{verbatim}
This driver program tests the distributed 
{\tt IVL\_MPI\_Bcast()} method.
Processor {\tt root} generates a random {\tt IVL} object
with {\tt nlist} lists.
The size of each list is bounded above by {\tt maxlistsize}.
Processor {\tt root} then sends its {\tt IVL} object to the other
processors.
Each processor computes a checksum for its object, and the error
are collected on processor 0.
Use the script file {\tt do\_IVL\_Bcast} for testing.
\par
\begin{itemize}
\item
The {\tt msglvl} parameter determines the amount of output.
Use {\tt msglvl = 1} for just timing output.
\item
The {\tt msgFile} parameter determines the message file --- if {\tt
msgFile} is {\tt stdout}, then the message file is {\it stdout},
otherwise a file is opened with {\it append} status to receive any
output data.
\item
The {\tt nlist} parameter specifies the number of lists.
\item
The {\tt maxlist} parameter is an upper bound on the size of each list.
\item
{\tt root} is the root processor for the broadcast.
\item
The {\tt seed} parameter is a random number seed.
\end{itemize}
%-----------------------------------------------------------------------
\item
\begin{verbatim}
testMMM msglvl msgFile nrowA ncolA nentA ncolX coordType
        inputMode symflag opflag seed real imag
\end{verbatim}
This driver program tests the distributed matrix-matrix multiply
$Y := Y + \alpha A X$,
$Y := Y + \alpha A^T X$ or
$Y := Y + \alpha A^H X$.
Process zero creates $Y$, $A$ and $X$ and computes
$Z = Y + \alpha A X$,
$Z = Y + \alpha A^T X$ or
$Z = Y + \alpha A^H X$.
Using random maps, it distributes $A$, $X$ and $Y$ among the other
processors.
The information structure is created using
{\tt MatMul\_MPI\_setup()}.
The local matrix $A^q$ is mapped to local coordinates.
The matrix-matrix multiply is computed,
and then all the $Y^q$ local matrices are gathered onto processor
zero into $Y$, which is then compared with $Z$ that was computed
using a serial matrix-matrix multiply.
The error is written to the message file by processor zero.
Use the script file {\tt do\_MMM} for testing.
\par
\begin{itemize}
\item
The {\tt msglvl} parameter determines the amount of output.
Use {\tt msglvl = 1} for just timing output.
\item
The {\tt msgFile} parameter determines the message file --- if {\tt
msgFile} is {\tt stdout}, then the message file is {\it stdout},
otherwise a file is opened with {\it append} status to receive any
output data.
\item
The {\tt nrowA} parameter is the number of rows in $A$.
\item
The {\tt ncolA} parameter is the number of columns in $A$.
\item
The {\tt nentA} parameter is the number of entries to be put into $A$.
\item
The {\tt nrowX} parameter is the number of rows in $X$.
\item
The {\tt coordType} parameter defines the coordinate type that will
be used during the redistribution. Valid values are {\tt 1} for rows,
{\tt 2} for columns and {\tt 3} for chevrons.
\item
The {\tt inputMode} parameter defines the mode of input.
Valid values are {\tt 1} for real entries 
and {\tt 2} for complex entries.
\item
The {\tt symflag} parameter specifies whether the matrix is
symmetric ({\tt symflag} = 0),
Hermitian ({\tt symflag} = 1) or
nonsymmetric ({\tt symflag} = 2)
\item
The {\tt opflag} parameter specifies the type of multiply,
{\tt 0} for $Y := Y + \alpha A X$,
{\tt 1} for $Y := Y + \alpha A^T X$ or
{\tt 2} for $Y := Y + \alpha A^H X$.
\item
The {\tt seed} parameter is a random number seed.
\item
The {\tt real} parameter is the real part of the scalar $\alpha$.
\item
The {\tt imag} parameter is the imaginary part of the scalar $\alpha$,
ignored for real entries.
\end{itemize}
%-----------------------------------------------------------------------
\item
\begin{verbatim}
testScatterDenseMtx msglvl msgFile nrow ncol inc1 inc2 seed root
\end{verbatim}
This driver program exercises the
{\tt DenseMtx\_MPI\_splitFromGlobalByRows()}
method to split or redistribute by rows
a {\tt DenseMtx} dense matrix object.
Process {\tt root} generates the {\tt DenseMtx} object.
A random map is generated (the same map on all processes) and the
object is redistributed using this random map.
The local matrices are then gathered into a second global matrix on
processor {\tt root} and the two are compared.
Use the script file {\tt do\_ScatterDenseMtx} for testing.
\par
\begin{itemize}
\item
The {\tt msglvl} parameter determines the amount of output.
Use {\tt msglvl = 1} for just timing output.
\item
The {\tt msgFile} parameter determines the message file --- if {\tt
msgFile} is {\tt stdout}, then the message file is {\it stdout},
otherwise a file is opened with {\it append} status to receive any
output data.
\item
The {\tt nrow} parameter is the number of rows for the matrix.
\item
The {\tt ncol} parameter is the number of columns for the matrix.
\item
The {\tt inc1} parameter is the row increment  for the matrix.
Valid values are {\tt 1} for column major and {\tt ncol} for row
major.
\item
The {\tt inc2} parameter is the column increment  for the matrix.
Valid values are {\tt 1} for row major and {\tt nrow} for column
major.
\item
The {\tt seed} parameter is a random number seed.
\item
The {\tt root} parameter is the root processor for the scatter and
gather.
\end{itemize}
%-----------------------------------------------------------------------
\item
\begin{verbatim}
testScatterInpMtx msglvl msgFile neqns seed 
                  coordType inputMode inInpMtxFile root
\end{verbatim}
This driver program tests the distributed 
{\tt InpMtx\_MPI\_splitFromGlobal()}
method to split a {\tt InpMtx} sparse matrix object.
Process {\tt root} reads in the {\tt InpMtx} object.
A random map is generated (the same map on all processes) and the
object is scattered from processor {\tt root} to the other processors.
Use the script file {\tt do\_ScatterInpMtx} for testing.
\par
\begin{itemize}
\item
The {\tt msglvl} parameter determines the amount of output.
Use {\tt msglvl = 1} for just timing output.
\item
The {\tt msgFile} parameter determines the message file --- if {\tt
msgFile} is {\tt stdout}, then the message file is {\it stdout},
otherwise a file is opened with {\it append} status to receive any
output data.
\item
The {\tt neqns} parameter is the number of equations for the matrix.
\item
The {\tt seed} parameter is a random number seed.
\item
The {\tt coordType} parameter defines the coordinate type that will
be used during the redistribution. Valid values are {\tt 1} for rows,
{\tt 2} for columns and {\tt 3} for chevrons.
\item
The {\tt inputMode} parameter defines the mode of input.
Valid values are {\tt 0} for indices only,
{\tt 1} for real entries 
and {\tt 2} for complex entries.
\item
The {\tt inInpMtxFile} parameter is the name of the file that
contain the {\tt InpMtx} object.
\end{itemize}
%-----------------------------------------------------------------------
\item
\begin{verbatim}
testSplitDenseMtx msglvl msgFile nrow ncol inc1 inc2 seed
\end{verbatim}
This driver program tests the distributed 
{\tt DenseMtx\_MPI\_splitByRows()}
method to split or redistribute by rows
a {\tt DenseMtx} dense matrix object.
Process zero generates the {\tt DenseMtx} object.
It is then split among the processes using a wrap map.
A random map is generated (the same map on all processes) and the
object is redistributed using this random map.
Use the script file {\tt do\_SplitDenseMtx} for testing.
\par
\begin{itemize}
\item
The {\tt msglvl} parameter determines the amount of output.
Use {\tt msglvl = 1} for just timing output.
\item
The {\tt msgFile} parameter determines the message file --- if {\tt
msgFile} is {\tt stdout}, then the message file is {\it stdout},
otherwise a file is opened with {\it append} status to receive any
output data.
\item
The {\tt nrow} parameter is the number of rows for the matrix.
\item
The {\tt ncol} parameter is the number of columns for the matrix.
\item
The {\tt inc1} parameter is the row increment  for the matrix.
Valid values are {\tt 1} for column major and {\tt ncol} for row
major.
\item
The {\tt inc2} parameter is the column increment  for the matrix.
Valid values are {\tt 1} for row major and {\tt nrow} for column
major.
\item
The {\tt seed} parameter is a random number seed.
\end{itemize}
%-----------------------------------------------------------------------
\item
\begin{verbatim}
testSplitInpMtx msglvl msgFile neqns seed coordType inputMode inInpMtxFile
\end{verbatim}
This driver program tests the distributed {\tt InpMtx\_MPI\_split()}
method to split or redistribute a {\tt InpMtx} sparse matrix object.
Process zero reads in the {\tt InpMtx} object.
It is then split among the processes using a wrap map.
A random map is generated (the same map on all processes) and the
object is redistributed using this random map.
Use the script file {\tt do\_SplitInpMtx} for testing.
\par
\begin{itemize}
\item
The {\tt msglvl} parameter determines the amount of output.
Use {\tt msglvl = 1} for just timing output.
\item
The {\tt msgFile} parameter determines the message file --- if {\tt
msgFile} is {\tt stdout}, then the message file is {\it stdout},
otherwise a file is opened with {\it append} status to receive any
output data.
\item
The {\tt neqns} parameter is the number of equations for the matrix.
\item
The {\tt seed} parameter is a random number seed.
\item
The {\tt coordType} parameter defines the coordinate type that will
be used during the redistribution. Valid values are {\tt 1} for rows,
{\tt 2} for columns and {\tt 3} for chevrons.
\item
The {\tt inputMode} parameter defines the mode of input.
Valid values are {\tt 0} for indices only,
{\tt 1} for real entries 
and {\tt 2} for complex entries.
\item
The {\tt inInpMtxFile} parameter is the name of the file that
contain the {\tt InpMtx} object.
\end{itemize}
%-----------------------------------------------------------------------
\item
\begin{verbatim}
testSymbFac msglvl msgFile inGraphFile inETreeFile seed 
\end{verbatim}
This driver program tests the distributed 
{\tt SymbFac\_MPI\_initFromInpMtx()} method
that forms a {\tt IVL} object that contains the necessary parts of
a symbolic factorization for each processor.
The program reads in the global {\tt Graph} and {\tt ETree}
objects.
Each processor creates a global {\tt InpMtx} object from the
structure of the graph and computes a global symbolic factorization
object using the serial {\tt SymbFac\_initFromInpMtx()} method.
The processors then compute a map from fronts to processors, and
each processor throws away the unowned matrix entries from the {\tt
InpMtx} object.
The processors then compute their necessary symbolic factorizations
in parallel.
For a check, they compare the two symbolic factorizations for error.
Use the script file {\tt do\_symbfac} for testing.
\par
\begin{itemize}
\item
The {\tt msglvl} parameter determines the amount of output.
Use {\tt msglvl = 1} for just timing output.
\item
The {\tt msgFile} parameter determines the message file --- if {\tt
msgFile} is {\tt stdout}, then the message file is {\it stdout},
otherwise a file is opened with {\it append} status to receive any
output data.
\item
The {\tt inGraphFile} parameter is the input file for the {\tt
Graph} object.
\item
The {\tt inETreeFile} parameter is the input file for the {\tt
ETree} object.
\item
The {\tt seed} parameter is a random number seed.
\end{itemize}
%-----------------------------------------------------------------------
\end{enumerate}
